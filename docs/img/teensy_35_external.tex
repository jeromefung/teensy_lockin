\documentclass[10pt]{standalone}
\usepackage{siunitx}
\usepackage{tikz}
\usetikzlibrary{calc}
\usepackage[american, siunitx]{circuitikz}

\listfiles

\begin{document}

\begin{circuitikz}[scale=0.72, transform shape]

\ctikzset{multipoles/dipchip/pin spacing=0.48}


\draw(0,0) node[dipchip, rotate = 90,
                          num pins = 70, hide numbers, external pins width=0](IC){
                          \rotatebox{-90}{Teensy 3.5}};

\node[above, font=\tiny] at (IC.bpin 1) {VIN};                       
\node[above, font=\tiny] at (IC.bpin 2) {GND};                       
\node[above, font=\tiny] at (IC.bpin 3) {3V3}; 
\node[above, font=\tiny] at (IC.bpin 4) {AREF}; 

% plain digital pins 0 - 13
\foreach \x in {5, ...,18} % x is the pin positions
   \pgfmathsetmacro \y{int(\x-5)}
    \node[above, font=\tiny] at (IC.bpin \x ) {\y};

% digital 14 - 23 are analog 0 - 9
\foreach \x in {19, ..., 28} % \x is the pin positions
     \pgfmathsetmacro \y {int(\x -5) } % digital name
     \pgfmathsetmacro \z {int (\x - 19)}  % analog name
     \node[above, font=\tiny, align=center, text width = 4pt] at (IC.bpin \x) {\y/ \\ A\z};

% plain digital pins 24 - 30
\foreach \x in {29, ..., 35} % x is the pin positions
   \pgfmathsetmacro \y{int(\x-5)}
    \node[above, font=\tiny] at (IC.bpin \x ) {\y};

% Pin 36 blank!!

\node[below, font=\tiny] at (IC.bpin 37) {A10}; 
\node[below, font=\tiny] at (IC.bpin 38) {A11}; 

% digital 31 - 39 are analog 12 - 20
\foreach \x in {39, ..., 47} % \x is the pin positions
     \pgfmathsetmacro \y {int(\x -8) } % digital name
     \pgfmathsetmacro \z {int (\x - 27)}  % analog name
     \node[below, font=\tiny, align=center, text width = 4pt] at (IC.bpin \x) {\y/ \\ A\z};

\node[below, font=\tiny, align=center, text width = 4pt] at (IC.bpin 48) {A \\ GND};
\node[below, font=\tiny, align=center, text width = 4pt] at (IC.bpin 49) {A21/ \\ DAC0} ;
\node[below, font=\tiny, align=center, text width = 4pt] at (IC.bpin 50) {A22/ \\ DAC1} ;


% plain digital pins 40 - 48
\foreach \x in {51, ..., 59} % x is the pin positions
   \pgfmathsetmacro \y{int(\x-11)}
    \node[below, font=\tiny] at (IC.bpin \x ) {\y};

\node[below, font=\tiny, align=center, text width = 4pt] at (IC.bpin 60) {49/ \\ A23} ;
\node[below, font=\tiny, align=center, text width = 4pt] at (IC.bpin 61) {50/ \\ A24} ;

% plain digital pins 51 - 57
\foreach \x in {62, ..., 68} % x is the pin positions
   \pgfmathsetmacro \y{int(\x-11)}
    \node[below, font=\tiny] at (IC.bpin \x ) {\y};
   
\node[below, font=\tiny, align=center, text width = 4pt] at (IC.bpin 69) {A25} ;
\node[below, font=\tiny, align=center, text width = 4pt] at (IC.bpin 70) {A26} ; 



\draw let \p{A} = (IC.bpin 2), \p{B} = (IC.bpin 18) in
  (IC.bpin 2)   to [short, -] ++(0, -1)
  to [esource, name = extRef] ++({(\x{B} - \x{A})}, 0) 
  -| (IC.bpin 18)
  
  (IC.bpin 2) ++ (0, -1) 
  to [short, -] ++(0, -1.5)
  to [capacitor, l_ = $\SI{10}{\pico\farad}$, font =\footnotesize] ++({(\x{B} - \x{A})}, 0) 
  -| (IC.bpin 18)
  ;

\tikzset{
  sqwave/.pic = {
  \draw[thick] (-0.2, -0.1) -- ++ (0, 0.2)  -- ++ (0.1, 0) -- ++(0, -0.2) --  ++(0.1, 0)  -- ++(0, 0.2) -- ++(0.1, 0) -- ++ (0, -0.2) -- ++ (0.1, 0) -- ++ (0, 0.2) -- ++(0.1, 0) ;
  }
}

\pic at (extRef) {sqwave};  
\node[yshift = -6mm, font = \footnotesize] at (extRef) {Reference signal}  ;

%\draw let \p{A} = (IC.bpin 37), \p{B} = (IC.bpin 38) in
%  (IC.bpin 38)   to [short, -] ++(0, 1)
%  to [esource, name = extRef] ++({(\x{A} - \x{B})}, 0) 
%  %-| (IC.bpin 38)
%  
%
%  ;

\draw (IC.bpin 38) to [short, -] ++(0, 0.25)
   to [short, -] ++ (-0.75, 0) 
   to [short, -] ++ (0, 0.75)
   to [short, -] ++ (0.25, 0)
   to [esource, name=intSig] ++(1,0)
   -| (IC.bpin 37);

\tikzset{
  signal/.pic = {
  \draw plot[smooth] file {noisy_sine.txt} ;
   }
}

\pic at (intSig) {signal};  
 
\node[yshift = 6mm, font=\footnotesize] at (intSig) {Signal of interest};  

\node[yshift = 15mm, xshift = -2mm, anchor=west, font=\large] at (IC.bpin 70) {(a) External reference} ;

\end{circuitikz}

\end{document}