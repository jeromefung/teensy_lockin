\documentclass[10pt]{standalone}
\usepackage{siunitx}
\usepackage{tikz}
\usetikzlibrary{calc}
\usepackage[american, siunitx]{circuitikz}

\listfiles

\begin{document}

\begin{circuitikz}[scale=0.72, transform shape]

\ctikzset{multipoles/dipchip/pin spacing=0.48}


\draw(0,0) node[dipchip, rotate = 90,
                          num pins = 28, hide numbers, external pins width=0](IC){
                          \rotatebox{-90}{Teensy 4.0}};

\node[above, font=\tiny] at (IC.bpin 1) {GND};                       

% plain digital pins 0 - 13
\foreach \x in {2, ...,14} % x is the pin positions
   \pgfmathsetmacro \y{int(\x-2)}
    \node[above, font=\tiny] at (IC.bpin \x ) {\y};

\node[below, font=\tiny] at (IC.bpin 15) {13}; 

\foreach \x in {16, ..., 25} % \x is the pin positions
     \pgfmathsetmacro \y {int(\x -2) } % digital name
     \pgfmathsetmacro \z {int (\x - 16)}  % analog name
     \node[below, font=\tiny, align=center, text width = 4pt] at (IC.bpin \x) {\y/ \\ A\z};

\node[below, font=\tiny] at (IC.bpin 26) {3.3 V};
\node[below, font=\tiny] at (IC.bpin 27) {GND};
\node[below, font=\tiny] at (IC.bpin 28) {Vin}; 
 
 


% plain digital pins 24 - 30
%\foreach \x in {29, ..., 35} % x is the pin positions
%   \pgfmathsetmacro \y{int(\x-5)}
%    \node[above, font=\tiny] at (IC.bpin \x ) {\y};

% Pin 36 blank!!

%\node[below, font=\tiny] at (IC.bpin 37) {A10}; 
%\node[below, font=\tiny] at (IC.bpin 38) {A11}; 

% digital 31 - 39 are analog 12 - 20
%\foreach \x in {39, ..., 47} % \x is the pin positions
%     \pgfmathsetmacro \y {int(\x -8) } % digital name
%     \pgfmathsetmacro \z {int (\x - 27)}  % analog name
%     \node[below, font=\tiny, align=center, text width = 4pt] at (IC.bpin \x) {\y/ \\ A\z};
%
%\node[below, font=\tiny, align=center, text width = 4pt] at (IC.bpin 48) {A \\ GND};
%\node[below, font=\tiny, align=center, text width = 4pt] at (IC.bpin 49) {A21/ \\ DAC0} ;
%\node[below, font=\tiny, align=center, text width = 4pt] at (IC.bpin 50) {A22/ \\ DAC1} ;
%
%
%% plain digital pins 40 - 48
%\foreach \x in {51, ..., 59} % x is the pin positions
%   \pgfmathsetmacro \y{int(\x-11)}
%    \node[below, font=\tiny] at (IC.bpin \x ) {\y};
%
%\node[below, font=\tiny, align=center, text width = 4pt] at (IC.bpin 60) {49/ \\ A23} ;
%\node[below, font=\tiny, align=center, text width = 4pt] at (IC.bpin 61) {50/ \\ A24} ;
%
%% plain digital pins 51 - 57
%\foreach \x in {62, ..., 68} % x is the pin positions
%   \pgfmathsetmacro \y{int(\x-11)}
%    \node[below, font=\tiny] at (IC.bpin \x ) {\y};
%   
%\node[below, font=\tiny, align=center, text width = 4pt] at (IC.bpin 69) {A25} ;
%\node[below, font=\tiny, align=center, text width = 4pt] at (IC.bpin 70) {A26} ; 



\draw let \p{A} = (IC.bpin 1), \p{B} = (IC.bpin 10) in
  (IC.bpin 1)   to [short, -] ++(0, -1)
  to [esource, name = extRef] ++({(\x{B} - \x{A})}, 0) 
  -| (IC.bpin 10)
  
%  (IC.bpin 1) ++ (0, -1) 
%  to [short, -] ++(0, -1.5)
%  to [capacitor, l_ = $\SI{10}{\pico\farad}$, font =\footnotesize] ++({(\x{B} - \x{A})}, 0) 
%  -| (IC.bpin 10)
  ;
%
\tikzset{
  sqwave/.pic = {
  \draw[thick] (-0.2, -0.1) -- ++ (0, 0.2)  -- ++ (0.1, 0) -- ++(0, -0.2) --  ++(0.1, 0)  -- ++(0, 0.2) -- ++(0.1, 0) -- ++ (0, -0.2) -- ++ (0.1, 0) -- ++ (0, 0.2) -- ++(0.1, 0) ;
  }
}
%
\pic at (extRef) {sqwave};  
\node[yshift = -6mm, font = \footnotesize] at (extRef) {Reference signal}  ;
%
%%\draw let \p{A} = (IC.bpin 37), \p{B} = (IC.bpin 38) in
%%  (IC.bpin 38)   to [short, -] ++(0, 1)
%%  to [esource, name = extRef] ++({(\x{A} - \x{B})}, 0) 
%%  %-| (IC.bpin 38)
%%  
%%
%%  ;
%

\draw let \p{A} = (IC.bpin 16), \p{B} = (IC.bpin 27) in
      (IC.bpin 27) to [short, -] ++(0, 0.75)
      to [esource, name=intSig] ++({(\x{A} - \x{B})},0)
   -| (IC.bpin 16);

\tikzset{
  signal/.pic = {
  \draw plot[smooth] file {noisy_sine.txt} ;
   }
}

\pic at (intSig) {signal};  
 
\node[yshift = 6mm, font=\footnotesize] at (intSig) {Signal of interest};  
%
%\node[yshift = 15mm, xshift = -2mm, anchor=west, font=\large] at (IC.bpin 70) {(a) External reference} ;

\end{circuitikz}

\end{document}